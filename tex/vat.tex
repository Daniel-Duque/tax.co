%% This is file `elsarticle-template-1-num.tex',
%%
%% Copyright 2009 Elsevier Ltd
%%
%% This file is part of the 'Elsarticle Bundle'.
%% ---------------------------------------------
%%
%% It may be distributed under the conditions of the LaTeX Project Public
%% License, either version 1.2 of this license or (at your option) any
%% later version.  The latest version of this license is in
%%    http://www.latex-project.org/lppl.txt
%% and version 1.2 or later is part of all distributions of LaTeX
%% version 1999/12/01 or later.
%%
%% Template article for Elsevier's document class `elsarticle'
%% with numbered style bibliographic references
%%
%% $Id: elsarticle-template-1-num.tex 149 2009-10-08 05:01:15Z rishi $
%% $URL: http://lenova.river-valley.com/svn/elsbst/trunk/elsarticle-template-1-num.tex $
%%
\documentclass[preprint,12pt]{elsarticle}

%% Use the option review to obtain double line spacing
%% \documentclass[preprint,review,12pt]{elsarticle}

%% Use the options 1p,twocolumn; 3p; 3p,twocolumn; 5p; or 5p,twocolumn
%% for a journal layout:
%% \documentclass[final,1p,times]{elsarticle}
%% \documentclass[final,1p,times,twocolumn]{elsarticle}
%% \documentclass[final,3p,times]{elsarticle}
%% \documentclass[final,3p,times,twocolumn]{elsarticle}
%% \documentclass[final,5p,times]{elsarticle}
%% \documentclass[final,5p,times,twocolumn]{elsarticle}


\usepackage[english]{babel}
\usepackage[utf8x]{inputenc}

%% The graphicx package provides the includegraphics command.
\usepackage{graphicx}
%% The amssymb package provides various useful mathematical symbols
\usepackage{amssymb}
%% The amsthm package provides extended theorem environments
%% \usepackage{amsthm}

%% This provides \toprule, among other stuff.
\usepackage{booktabs}

%% To include files with names with weird characters like space and dot.
\usepackage{grffile}

%% natbib.sty is loaded by default. However, natbib options can be
%% provided with \biboptions{...} command. Following options are
%% valid:

%%   round  -  round parentheses are used (default)
%%   square -  square brackets are used   [option]
%%   curly  -  curly braces are used      {option}
%%   angle  -  angle brackets are used    <option>
%%   semicolon  -  multiple citations separated by semi-colon
%%   colon  - same as semicolon, an earlier confusion
%%   comma  -  separated by comma
%%   numbers-  selects numerical citations
%%   super  -  numerical citations as superscripts
%%   sort   -  sorts multiple citations according to order in ref. list
%%   sort&compress   -  like sort, but also compresses numerical citations
%%   compress - compresses without sorting
%%
%% \biboptions{comma,round}

% \biboptions{}


\journal{Journal Name}

\begin{document}

\begin{frontmatter}

%% Title, authors and addresses

\title{The Colombian VAT's Demographic Incidence}

%% use the tnoteref command within \title for footnotes;
%% use the tnotetext command for the associated footnote;
%% use the fnref command within \author or \address for footnotes;
%% use the fntext command for the associated footnote;
%% use the corref command within \author for corresponding author footnotes;
%% use the cortext command for the associated footnote;
%% use the ead command for the email address,
%% and the form \ead[url] for the home page:
%%
%% \title{Title\tnoteref{label1}}
%% \tnotetext[label1]{}
%% \author{Name\corref{cor1}\fnref{label2}}
%% \ead{email address}
%% \ead[url]{home page}
%% \fntext[label2]{}
%% \cortext[cor1]{}
%% \address{Address\fnref{label3}}
%% \fntext[label3]{}


%% use optional labels to link authors explicitly to addresses:
%% \author[label1,label2]{<author name>}
%% \address[label1]{<address>}
%% \address[label2]{<address>}

\author{Jeff Brown and David Suárez Castellanos}

\address{Bogotá, Colombia}

\begin{abstract}
%% Text of abstract
The VAT rate on goods varies between 0\% and 19\%. So does the fraction of total income an individual or a household pays toward it.
\end{abstract}

\begin{keyword}
Colombia \sep Value Added Tax \sep Tax Incidence
\end{keyword}

\end{frontmatter}


\section{What is the VAT?}

A VAT is a tax levied on purchases of goods and services. The Colombian government receives revenue from many other taxes, too -- the individual and corporate income taxes, real estate taxes, taxes on alcohol, tobacco, and gas, tariffs ... However the VAT is its single biggest revenue source. The VAT goes entirely to the federal government, which derives 1/3 of revenues from it. Federal tax revenue is 80\% of total tax revenue in Colombia.
\\


\subsection{What fraction of the price is the VAT?}

The default level of the VAT is 19\%. There are numerous exemptions, however; the tax code singles out 310 categories of goods and services, some quite broad, for special VAT treatment. Some of these exempted items have a 5\% VAT; the rest, 0\%.

A flat VAT falls more heavily on poorer households, because these spend a greater fraction of their income. A VAT can be made more progressive by taxing luxury items at a higher rate than necessities, since poorer households consume fewer luxury goods. Indeed, some of Colombia's VAT exclusions appear intended to relieve the tax burden on lower income households: Bus rides, rent on a home, water, sewerage, and phone services, medical goods and services, and hundreds of kinds of food are subject to zero VAT. Other zero-VAT goods and services include payments associated with a second home, ocean travel, banking services, political contributions, and "espectaculos" such as ballet, cinema, theater, sporting events, circuses, and fairs.


\section{How we determined which populations carry how much of the VAT burden}

\subsection{The Encuesta Nacional de Presupuestos de Hogares (ENPH)}

We used the ENPH, a survey of Colombian households conducted over 2016 and 2017. It picks a representative sample of homes, and then surveys everyone in the home, dividing them into "households". A household's members consist of those people who eat and sleep there daily.

The ENPH is a big survey. We selected only a fraction of the data collected in it for use in this analysis. Those selections include:

\begin{itemize}
  \item Data on goods and services: Each good or service the ENPH asks households about is identified by a COICOP code and a verbal description.
  \item  Demographic data: Age, sex, race, education level, literacy, employment, and income.
  \item  Purchase data
  \begin{itemize}
      \item  What someone bought (as identified by its COICOP code)
      \item  How many
      \item  How often
      \item  What they spent on it
  \end{itemize}
\end{itemize}


\subsection{Our COICOP-VAT match}

Our other source of data is something we constructed ourselves. It associates a tax rate -- 0, 5 or 19\% -- to each of the COICOP codes, based on the tax laws and the description of goods and services from the ENPH.

\section{What the data reveal}

\subsection{What purchases look like}

The survey records just under 7.5 million purchases. When people buy something, the median number they buy is 2. The maximum was a million, and the minimum -70 (yes, negative seventy).
\begin{figure}[t]
\centering\includegraphics[width=1\linewidth]{../output/vat-pics/purchases/quantity.png}
\end{figure}

The survey includes a few categories for frequency of purchase. We translated those into numerical frequencies. A frequency of 30 indicates someone makes the same purchase 30 times in a month. That is the highest value. The least is .0027 -- once every two years.
\begin{figure}[t]
\centering\includegraphics[width=1\linewidth]{../output/vat-pics/purchases/frequency.png}
\end{figure}

Multiplying the value of a purchase by its frequency results in the amount of money someone spends on that good in a month, on average. It ranges from 0 to 23 million pesos, with an average of 34,000 and a median of 16,000.
\begin{figure}[t]
\centering\includegraphics[width=1\linewidth]{../output/vat-pics/purchases/value.png}
\end{figure}

The VAT paid on a purchase ranges from 0 to 4.4 million pesos. The average is 700 pesos. More than 80\% of all purchases carry zero VAT -- which is plausible, given how many common purchases are exempt.
\begin{figure}[t]
\centering\includegraphics[width=1\linewidth]{../output/vat-pics/purchases/vat in pesos.png}
\end{figure}


\section{What individuals look like}

The 300,000 Colombians surveyed in the ENPH range in age from 0 to 110. Half are under 30. In education, only 5\% indicate having less than a primary education; 45\% indicate having completed primary or secondary and no further; 23\% indicate "media", and 22\% indicate having a university education.
\begin{figure}[t]
\centering\includegraphics[width=1\linewidth]{../output/vat-pics/people/age.png}
\end{figure}

53\% of respondents are female. 30\% of respondents are students. 

Incomes range from 0 to 80 million pesos per month, with a median of 800,000 -- just over the minimum wage. The number of transactions per person range from 1 to 144. [[Given that the minimum age is 0, and that 93\% of purchases are household-communal (for the three data sets that include such a variable), the positive minimum number of purchases seems strange.]]
\begin{figure}[t]
\centering\includegraphics[width=1\linewidth]{../output/vat-pics/people/income.png}
\end{figure}

\begin{figure}[t]
\centering\includegraphics[width=1\linewidth]{../output/vat-pics/people/transactions per month.png}
\end{figure}


\subsection{What households look like}

The average household has 3.37 members. The biggest has 22. Nearly 15\% of households are individuals living alone.
\begin{figure}[t]
\centering\includegraphics[width=1\linewidth]{../output/vat-pics/households/size.png}
\end{figure}

The average number of purchases made by a household is 44 in a month; the minimum observed number is 1, and the maximum 358.
\begin{figure}[t]
\centering\includegraphics[width=1\linewidth]{../output/vat-pics/households/transactions per month.png}
\end{figure}

The age of the oldest person in a household ranges from 0 [[who are they?]] to 110. Half of households contain somebody over fifty. A quarter of households include someone under 5 years of age, and half include someone under 14.
\begin{figure}[t]
\centering\includegraphics[width=1\linewidth]{../output/vat-pics/households/youngest.png}
\end{figure}
\begin{figure}[t]
\centering\includegraphics[width=1\linewidth]{../output/vat-pics/households/oldest.png}
\end{figure}

More than 3/4 of households include someone who has attained at least a "media" educational degree.
\begin{figure}[t]
\centering\includegraphics[width=1\linewidth]{../output/vat-pics/households/transactions per month.png}
\end{figure}

90\% of households include at least one female member, and 86\% at least one male.

The minimum and maximum values for household income are little changed from their values for individuals; however the median is substantially higher -- 1.02 million as opposed to 800,000 for individuals. That difference seems likely to reflect the prevalence of households with one full-time earner and another part-time earner.
\begin{figure}[t]
\centering\includegraphics[width=1\linewidth]{../output/vat-pics/households/income.png}
\end{figure}



%%% jbb-end


\section{Flotsam and Jetsam}

\begin{tabular}{lrrr}
\toprule
{} &  count &        min &         max \\
income-decile &        &            &             \\
\midrule
0             &   4785 &       98.0 &    450000.0 \\
1             &   5086 &   450098.0 &    689454.0 \\
2             &   4982 &   689455.0 &    737717.0 \\
3             &   4300 &   737718.0 &    840000.0 \\
4             &   4794 &   840098.0 &   1050000.0 \\
5             &   4657 &  1052000.0 &   1378000.0 \\
6             &   4760 &  1378313.0 &   1610000.0 \\
7             &   4799 &  1611000.0 &   2100000.0 \\
8             &   4734 &  2100098.0 &   3050000.0 \\
9             &   4756 &  3052000.0 &  83541600.0 \\
\bottomrule
\end{tabular}

\\

\begin{figure}[h]
\centering\includegraphics[width=1\linewidth]{../output/vat-pics/VAT over consumption by income decile.png}
\caption{Figure caption}
\end{figure}

\begin{tabular}{lrrr}
\toprule
{} &  count &        min &         max \\
income-decile &        &            &             \\
\midrule
0             &   4785 &       98.0 &    450000.0 \\
1             &   5086 &   450098.0 &    689454.0 \\
2             &   4982 &   689455.0 &    737717.0 \\
3             &   4300 &   737718.0 &    840000.0 \\
4             &   4794 &   840098.0 &   1050000.0 \\
5             &   4657 &  1052000.0 &   1378000.0 \\
6             &   4760 &  1378313.0 &   1610000.0 \\
7             &   4799 &  1611000.0 &   2100000.0 \\
8             &   4734 &  2100098.0 &   3050000.0 \\
9             &   4756 &  3052000.0 &  83541600.0 \\
\bottomrule
\end{tabular}

\\

\end{document}
